\documentclass[final,leqno]{siamltex}

% definitions used by included articles, reproduced here for 
% educational benefit, and to minimize alterations needed to be made
% in developing this sample file.

\newcommand{\pe}{\psi}
\def\d{\delta} 
\def\ds{\displaystyle} 
\def\e{{\epsilon}} 
\def\eb{\bar{\eta}}  
\def\enorm#1{\|#1\|_2} 
\def\Fp{F^\prime}  
\def\fishpack{{FISHPACK}} 
\def\fortran{{FORTRAN}} 
\def\gmres{{GMRES}} 
\def\gmresm{{\rm GMRES($m$)}} 
\def\Kc{{\cal K}} 
\def\norm#1{\|#1\|} 
\def\wb{{\bar w}} 
\def\zb{{\bar z}} 

% some definitions of bold math italics to make typing easier.
% They are used in the corollary.

\def\bfE{\mbox{\boldmath$E$}}
\def\bfG{\mbox{\boldmath$G$}}

\usepackage{amsfonts}

\title{A minimal non-empty set that is arbitrarily controlled by me within physical restrictions}

% The thanks line in the title should be filled in if there is
% any support acknowledgement for the overall work to be included
% This \thanks is also used for the received by date info, but
% authors are not expected to provide this.

\author{Peifeng Wang\thanks{Guanghua Road 1\#, 34-1-3-5, Yanta District, Xi'an, Shaanxi, P. R. China 710075 ({\tt peifeng\_w@yahoo.com})}}

\begin{document}

\maketitle

\begin{abstract}
In an effort to understand mind, a definition of the concept ``I" is presented in a mathematically sound and physical implementation friendly way.
\end{abstract}

\begin{keywords} 
Mind, Soul, Consciousness
\end{keywords}

%\begin{AMS}

%\end{AMS}

\pagestyle{myheadings}
\thispagestyle{plain}
\markboth{}{}

Mind, soul and consciousness are curious topics which have been studied in various disciplines, from biological mechanism to computer emulation of intelligent entities. While each of the fields has seen independent progress, it is believed that a single model underlies the multiple facets.

In an exploration of a unified model, it is noted that ``I" represents a core component of mind, soul and consciousness. In daily word, ``I" have a mind, conversely, a mind uses ``I" while referring to itself as if the mind has ``I". Thus an (intuitive) equivalence relationship can be formed between a mind and its associated ``I"\cite{intuition}. Due to that equivalence, the following discussion presents a definition of the concept of ``I" in the realm of science and mathematics.

\begin{definition} 
\label{def:I}
I is a minimal non-empty set that is arbitrarily controlled by I within physical restrictions
\end{definition}

While definition \ref{def:I} appears a poorly constructed circular definition with grammar errors at a first sight, it can be transformed into a mathematical formula and shed light on the physical realization.

To obtain a mathematical representation of definition \ref{def:I}, let $W$ be the set of all objects in the physical world, $\mathbb{S}$ be the set of all subset of $W$, i.e. $\mathbb{S}=\{s|s\subset W\}$. Then a function $f$ is defined to represent a physical relationship between elements of $\mathbb{S}$, i.e. subsets of $W$.

\begin{definition}
\label{def:f_control}
$f:\mathbb{S}\rightarrow\mathbb{S}$ such that for $A\in\mathbb{S}$, $B=f(A)$ is the set that is arbitrarily controlled by $A$.
\end{definition}

Definition \ref{def:f_control} is not complete as ``arbitrarily" and ``control" are not defined. The term ``arbitrarily" is intended to incorporate ``free will", which, by definition, is capable of proposing arbitrarily unlimited possibilities of axioms. In real world, ``free will" is manifested by interaction with other objects.  which is dealt with by the term ``control". It is not attempted to completely specify the terms to prescribe ``free will", instead, further clarification will be presented later.

While definition \ref{def:f_control} is left incomplete in order to capture the character of ``free will", it can still be utilized within a mathematical formula, the following equation
\begin{eqnarray}
\label{eqn:I_norestriction}
I_0=f(I_0)
\end{eqnarray}
can be constructed and is read as

\emph{$I_0$ is a set that is arbitrarily controlled by $I_0$}

Then $I_0$ is the unrestricted version of definition \ref{def:I}. To apply physical restrictions, the set of physical restrictions is defined as
\begin{eqnarray}
P=\{p|p ~ is ~ prohibited ~ in ~ physical ~ world\}
\end{eqnarray}
then the set of physically realizable object is the complement of $P$, i.e. $\bar P$
\begin{eqnarray}
\bar P=\{p|p ~ is ~ realizable ~ in ~ physical ~ world\}
\end{eqnarray}
thus within physical restrictions, equation (\ref{eqn:I_norestriction}) becomes
\begin{eqnarray}
\label{eqn:I}
I=f(I)\cap \bar P=f(I)\setminus P=f(I)-P
\end{eqnarray}

With definition \ref{def:f_control}, equation (\ref{eqn:I}) is read as

\emph{$I$ is a set that is arbitrarily controlled by $I$ within physical restrictions}

When ``I" is required to be minimal and non-empty, equation (\ref{eqn:I}) translates exactly into definition \ref{def:I}. Alternatively, ``I" is a minimal non-empty solution to equation (\ref{eqn:I}).

With equation (\ref{eqn:I}) representing definition \ref{def:I}, various parts of definition \ref{def:I} are examined.

``I" in the definition is a mathematical entity, so ``I is" is used rather than ``I am", likewise, ``by I" is used instead of ``by me". However, the definition of mathematical entity ``I" is intended to approximate the ``I" of a mind in physical reality as in ``I am".

``I" is defined as ``a [...] set" because of its bounded nature. Something belong to ``I", other things belong to other minds. Things belonging to ``I" are not part of other minds, while things belonging to other minds are not part of ``I". This things-belong-to-mind relationship can be specified well by ``a set" and its elements.

In real world, the empty set $\emptyset$ represents nothing. It is proper to expect that, nothing controls nothing, or in the form of equation (\ref{eqn:I_norestriction}), $\emptyset=f(\emptyset)$. It then follows
\begin{eqnarray}
\emptyset=f(\emptyset)\cap \bar P
\end{eqnarray}
thus $\emptyset$ satisfies equation (\ref{eqn:I}). As $\emptyset$ is only a trivial case, it is excluded by the requirement of ``non-empty" set.

It is also plausible that for 2 disjoint sets satisfying equation (\ref{eqn:I}), their union also satisfies equation (\ref{eqn:I}). \begin{eqnarray}
I_1\cap I_2=\emptyset, I_1=f(I_1)\cap \bar P, I_2=f(I_2)\cap \bar P \rightarrow I_1\cup I_2=f(I_1\cup I_2)\cap \bar P
\end{eqnarray}
This indicates that, in addition to individual mind (I), a collection of minds (we) also satisfies equation (\ref{eqn:I}). By intuition, ``I" is supposed to be indivisible. To do so, a partial ordering can be defined on $\mathbb{S}$,
\begin{eqnarray}
\label{eqn:ordering}
A\subset B \leftrightarrow A\prec B
\end{eqnarray}
then by defining the set to be ``minimal", it is indivisible.

Just as in definition \ref{def:f_control}, ``arbitrarily" in definition \ref{def:I} captures the character of ``free will". In a spirit similar to Godel's theorem\cite{Godel}, one can argue that, if ``arbitrariness" was defined in some formal system, an agent with ``free will" could act arbitrarily to deliberately contradict the prediction of that formal system so as to violate the definition of ``arbitrariness". The violation does not arise in any formal logic way, rather, it is from ``arbitrariness", the exact term to be defined. As a result, there can not be a mathematically complete way to define "free will". ``arbitrariness" reveals the mathematically unpredictable (undecidable) nature of the subject, and means mathematically unrestricted while physically restricted. In another word, a mind is not mathematically represented by a finite set of axioms and inference rules, though it can still be physically deterministic.

``control" specifies interactions within physical world. Physical objects have states. For 2 objects in physical world $W$, i.e. $x\in W, y\in W$, if variations of state of $x$ can affect the state of $y$,  a ``control" relationship is formed as ``$x$ controls $y$"\cite{light}, and is noted as $ctrl(x,y)$.  The ``control" relationship can be extended to sets of objects. For $A\in\mathbb{S},B\in\mathbb{S}$, which are subsets of objects in the physical world $W$, $B=f(A)$ as in definition \ref{def:f_control} represents ``$A$ controls $B$". More specifically, it means that, for all $y\in B$, there exists $x\in A$ such that $x$ controls $y$, or in notation $\forall y\in B.\exists x\in A.ctrl(x,y)$.

``within physical restrictions" explicitly specify the applicable realm of definition \ref{def:I}, as opposite to some abstract spiritual domain. One notable physical restriction is that in reality, it takes time for any computation. So there is latency in $f$, which makes equation (\ref{eqn:I}) dynamic rather than static.

In the real world, equation (\ref{eqn:I}) has plenty of solutions, among which many are minimal according to the ordering in equation (\ref{eqn:ordering}). Those many minimal solutions of equation (\ref{eqn:I}) corresponds to the respective ``I" of the many existing minds.

It should be realized that circular definition is practicable, for example, ``x is a number that equals to the square of x" is a well formed statement and equivalent to $x=x^2$. Circular definition just opens a way to form an equation. The self referential approach in definition \ref{def:I} is beneficial to its purpose, as when a mind refers to itself, the concept ``I" arises.

%\section*{Acknowledgments}
 
 
\begin{thebibliography}{10} 
\bibitem{intuition}Intuition is used here because of lack of a formal definition of mind.
\bibitem{Godel}Raatikainen, Panu, "Gödel's Incompleteness Theorems", The Stanford Encyclopedia of Philosophy (Fall 2018 Edition), Edward N. Zalta (ed.), URL = https://plato.stanford.edu/archives/fall2018/entries/goedel-incompleteness/  
\bibitem{light}e.g. for a light bulb connected to a switch, changes of the state of the switch (on/off) affect the state of the light bulb (bright/dark), it is then said, the switch controls the light bulb.
\end{thebibliography} 

\end{document} 

